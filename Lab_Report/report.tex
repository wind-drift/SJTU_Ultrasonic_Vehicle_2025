\documentclass[12pt,a4paper]{article}
\usepackage{ctex}
\usepackage{geometry}
\usepackage{graphicx}
\usepackage{amsmath}
\usepackage{booktabs}
\usepackage{float}
\usepackage{listings}
\graphicspath{{image/}}
\renewcommand{\figurename}{图}
\usepackage{xcolor}

\lstset{
    language=C++,
    basicstyle=\ttfamily\small,
    keywordstyle=\color{blue},
    commentstyle=\color{gray},
    stringstyle=\color{orange},
    numbers=left,
    numberstyle=\tiny,
    breaklines=true,
    frame=single
}

\geometry{left=2.5cm,right=2.5cm,top=2.5cm,bottom=2.5cm}

\title{基于超声波定位的自动跟随小车设计}
\author{指导老师: \quad 作者:}
\date{\today}

\begin{document}
\maketitle

\begin{table}[H]
\centering
\caption{作者信息}
\begin{tabular}{lll}
\toprule
姓名 & 学号 & 分工 \\ 
\midrule
XXX & 1234567890 & XXXX \\ 
XXX & 1234567890 & XXXX\\
XXX & 1234567890 & XXXX \\ 
\bottomrule
\end{tabular}
\end{table}

\tableofcontents
\newpage

%==========================
\section{实验目的}

本实验基于数字电子技术、模拟电子技术与嵌入式系统基础知识,设计并实现基于超声波信号的自动跟随小车系统。
通过自行搭建超声波发送与接收电路,并结合单片机进行信号处理与控制,
实现小车对超声波声源的自动定位与跟随,若无超声波信号,则小车自动进入旋转搜索模式,等待捕获信号源。

通过本实验,可以加深对超声波传感技术、信号调理电路设计以及嵌入式控制系统开发的理解与应用能力。

%==========================
\section{系统总体方案设计}

\subsection{系统结构}

系统结构主要分为三部分:超声波发送端、超声波接收端、电机驱动部分。

发送端通过6.5V电池组模块供电,由单片机信号驱动,产生频率 $40kHz$ 的超声波信号

接收端含电机驱动模块、接收器滤波整形电路以及 MCU 。
接收器安装于小车前方两侧接收声波信号,通过滤波整形电路将正弦信号转换为40kHz的脉冲信号,输入到 MCU 的 ADC 引脚进行信号处理。

\subsection{工作原理}

由于发送端与接收端分离,系统无法直接获取发送时刻,因此采用双接收器时间差定位方法。
通过比较左右接收器接收到超声波信号的时间差,判断声源相对于小车的方向,结合PID控制算法控制电机驱动模块,从而控制小车进行转向与跟随。

当系统检测不到有效超声信号时,小车进入搜索模式,原地旋转以重新寻找信号源。

%==========================
\section{硬件系统设计}

\subsection{超声波发送电路}

超声波发射端用于产生稳定的高频超声信号,为接收端提供连续的声源参考。
本系统采用独立供电的手持式超声波发送模块,与小车接收端相互分离,符合实验任务书对系统结构的要求。

实验中采用 MAX232 芯片构成的电平转换电路,将控制器输出的 PWM 信号转换为幅值更高的差分激励信号,
从而有效驱动超声波发射器,提高超声波信号的发射强度和传播距离。

在工作过程中,控制器持续输出若干周期的 $40kHz$ 脉冲信号,超声波发射器在电激励作用下产生连续的超声波信号并向空间辐射。
接收端通过检测该连续超声波信号的到达时间差,实现对声源方向的判断与跟随控制。

\subsection{超声波接收滤波与整形电路}

\begin{figure}[h]
	\centering
	\includegraphics[scale=0.3]{滤波电路.png}
	\caption{滤波电路}
	\label{fig:滤波}
     \end{figure}
         
     该电路为单电源($+5\text{ V}$)条件下的超声波接收前端放大电路,其主要功能包括:
     \begin{itemize}
     	\item 对超声波接收信号进行交流耦合
     	\item 抑制直流与低频噪声
     	\item 在单电源供电下实现交流信号放大
     	\item 通过 $2.5\text{ V}$ 偏置实现虚拟地
     \end{itemize}
     
     需要注意的是,实际运用中该电路会产生自激振荡,故需确保超声波接受器在无信号时$\text{Ui}$引脚与地之间保持短接状态。
    
  
     输出信号的频率等于超声波接收器所接收到的信号频率(通常为 $40\,\text{kHz}$)。
     输入端由电阻 $R$ 与电容 $C$ 构成一阶 RC 高通滤波器,其截止频率为:
     \begin{equation}
     	f_c = \frac{1}{2\pi RC}
     \end{equation}
     代入参数:
     \[
     R = 5.1\,\text{k}\Omega, \quad
     C = 620\,\text{pF}
     \]
     \begin{equation}
     	f_c = \frac{1}{2\pi \times 5100 \times 620 \times 10^{-12}}
     	\approx 5.0 \times 10^{4}\,\text{Hz}
     \end{equation}
     \noindent
     因此,该电路允许约 $50\,\text{kHz}$ 以上的信号通过,对低频信号具有明显抑制作用,适合用于 $40\sim60\,\text{kHz}$ 的超声波接收应用。
     
   \begin{figure}[ht]
   	\centering
   	\includegraphics[scale=0.25]{2.5v稳压电路.png}
   	\caption{2.5v偏置电路}
   	\label{fig:2.5v}
   	
   \end{figure}
     由于小车为单电源供电,故采用如图2所示的2.5V稳压管提供2.5V的偏置电压,
     该电压作为运算放大器的参考电位,并通过 RC 网络进行滤波以提高稳定性。
     由于运算放大器采用单电源供电($0\sim5\,\text{V}$),无法直接处理负电压信号,因此引入 $2.5\,\text{V}$ 作为交流意义上的参考地(虚拟地)。
     设原始超声波信号为:
     \[
     v_{\text{ac}}(t) = \pm V_m
     \]
     
     加入偏置后,运放输入信号变为:
     \begin{equation}
     	v_{\text{in}}(t) = 2.5\,\text{V} \pm V_m
     \end{equation}
     
     从而保证信号始终处于运放允许的输入范围内,避免削波失真。
     偏置后的输出信号形式为:
     \begin{equation}
     	v_{\text{out}}(t) = 2.5\,\text{V} \pm A_v V_m
     \end{equation}
      \begin{figure}[ht]
     	\centering
     	\includegraphics[scale=0.3]{比较脉冲电路.png}
     	\caption{放大电路}
     	\label{fig:compare_pause}
     \end{figure}
     
     
     实际运用中发现 $V_m$可达$1\text{V}$以上,利用图3中的单限比较器可实现方波信号的产生。由于滤波电路输出的偏置高于2.5V,并且由于LM413运放性能受限,上升沿到达最高前便下降,故滤波输出接入运放$N$端,确保晶体管在无信号时保持高电平,接收到信号时产生频率为$40\text{Hz}$的脉冲信号。
     
  
     
\subsection{单片机与电机驱动电路}

本系统选用 \texttt{ESP32} 单片机作为核心控制器,负责超声波信号的采集、时间差计算、控制算法运算以及电机驱动信号的输出。

在超声波信号采集方面,左右两个超声波接收电路输出经滤波与整形后的 $40kHz$ 数字脉冲信号,分别接入 \texttt{ESP32} 的外部中断引脚。
单片机在检测到信号上升沿时记录当前时间,通过比较左右信号的到达时间差判断声源相对于小车的方向。

小车的运动控制采用两轮差速驱动方式。电机驱动电路中,每个电机由单路 PWM 信号控制,其方向端在硬件上固定为正转状态。
单片机通过改变 PWM 信号的占空比调节电机转速,从而实现小车的速度控制与转向控制。

在具体控制策略上,系统将小车的运动分解为基础前进速度与转向修正量两部分。
基础前进速度用于保证小车持续向前运动,转向修正量由控制算法计算得到,并以速度差分量的形式叠加至左右电机的驱动信号中。
通过增大一侧电机转速、减小另一侧电机转速的方式实现转向控制。

当系统未检测到有效的超声波信号时,单片机控制左右电机以不同转速运行,
使小车围绕单侧车轮缓慢旋转,从而进入搜索状态以重新捕获超声波信号。

%==========================
\section{软件系统设计}

为提高系统的可读性、可维护性与调试效率,本系统的软件部分采用模块化与面向对象相结合的设计方法。

软件系统主要由以下功能模块构成,各模块之间通过清晰的接口进行通信与协作:
\begin{itemize}
    \item 主控制模块:\texttt{main\_controller},负责系统状态管理与各功能模块的协调调度;
    \item 超声波检测模块:\texttt{ultrasonic\_controller},用于采集左右超声波信号并计算时间差;
    \item PID 控制模块:\texttt{pid\_controller},根据时间差计算转向控制量;
    \item 电机控制模块:\texttt{motor\_controller},根据控制指令输出电机驱动信号。
\end{itemize}


\subsection{软件系统总体结构}

系统程序以 \texttt{Arduino} 框架为基础,主函数仅负责系统初始化与循环调度。
各功能模块在初始化阶段完成硬件配置,在主循环中由主控制模块统一调度执行。

\subsection{主控制模块设计}

主控制模块为系统的软件核心,负责协调各功能模块的运行,其主要功能包括:
超声波信号状态判断、时间差数据获取、控制算法调用以及系统状态切换。
系统采用简单的状态机结构,定义跟随状态 \texttt{FOLLOWING} 与搜索状态 \texttt{SEARCHING} 两种工作模式。

在主循环中,控制器首先判断超声波检测模块是否获得了一次完整的左右信号检测。
当检测到有效信号时,系统进入跟随状态;
若在一定时间内未检测到有效信号,则系统自动切换至搜索状态。

\subsection{超声波时间差检测模块}

超声波检测模块基于外部中断方式实现。
左右两个超声波接收通道分别接入单片机的外部中断引脚,当检测到经整形后的超声波脉冲信号下降沿时,
通过微秒级定时函数记录当前时间。为避免多次触发干扰,每个接收通道在完成一次检测后暂时锁定,直至左右信号均被接收且超声波发生器处于静默期。

当左右接收器均完成一次有效触发后,系统计算左右信号的到达时间差,并时间差映射到\texttt{q15\_t},作为后续方向控制的输入量。
为增强系统的稳定性,对时间差数据设置限幅,避免异常信号对控制过程产生过大影响。

\subsection{PID 方向控制算法}

系统采用经典 PID 控制算法对小车的转向进行控制。
控制目标为使左右超声波接收信号的到达时间差趋近于零,即小车正对超声波声源方向。
将映射后的时间差变量再次映射到 $[-1,1]$ 区间, PID 控制器将其作为反馈量,输出运算结果。
运算结果经算法低通滤波后,根据 \texttt{analogWrite()} 函数输入范围,线性放大为电机控制量。

在实现中, PID 控制器采用位置式算法结构,并引入积分限幅与输出限幅机制,以防止积分饱和和控制输出过大。
PID 参数通过实验整定获得,在实验运行速度范围内能够保证系统稳定运行。

\subsection{电机控制与运动策略}

电机控制模块采用两轮差速控制策略。
系统将小车的运动分解为基础前进速度与转向修正量两部分,其中基础速度用于保证小车持续向前运动,
转向修正量由 PID 控制器计算得到,并以速度差分的形式叠加至左右电机。

在驱动方式上,电机仅支持单方向转动,转速由单路 PWM 信号控制。
通过提高一侧电机转速并降低另一侧电机转速,实现小车的转向控制。

为提高系统稳定性,电机 \texttt{PWM} 信号的更新频率受到时间间隔限制,避免因控制量快速变化导致电机频繁调整,从而提升整体运行平滑性。

\subsection{搜索模式与状态切换机制}

当系统在设定时间内未检测到有效超声波信号时,主控制模块判定信号丢失,系统自动进入搜索状态。
在搜索状态下,小车通过单侧驱动方式缓慢转向,以扩大超声波信号的搜索范围。
当重新检测到有效信号后,系统立即切换回跟随状态,恢复正常跟随控制。

\section{遇到的问题及解决方案}

\subsection{电控设计}

在超声波控制器的调试过程中,发现接收端采集的数据中存在大量数值极小的异常时间差,
同时仅夹杂少量符合预期的有效数据。

经分析认为,该现象源于单片机在超声波发射期间持续接收超声波信号。
由于超声波为连续周期信号,在每个发射周期内,左右接收通道均会多次检测到相位一致的脉冲信号。
其中,仅首次接收到的两路信号能够真实反映声波到达时间差,而后续接收到的脉冲由于相位关系固定,
其计算得到的时间差不具有实际物理意义,从而表现为无效数据。

针对上述问题,系统在软件中引入接收锁定与静默机制:当某一通道首次检测到超声波信号后,
立即锁定该通道,禁止后续触发;待左右两通道均完成一次有效接收并完成时间差计算后,
系统进入短暂静默状态,直至超声波发射器结束当前发射周期。
通过该方法,有效消除了重复触发导致的无效数据,提高了时间差测量的稳定性与可靠性。

\subsection{电路设计}

在超声波接收滤波与整形电路的调试过程中,发现接收前端在无超声信号输入时易产生自激振荡现象,
导致输出端出现非期望的周期性波形,影响后续信号判别的稳定性。

经分析认为,该问题主要源于单电源运算放大器在交流放大条件下对输入偏置和输入阻抗较为敏感,
当超声波接收器处于悬空或高阻状态时,输入端易受到噪声和偏置电压扰动,从而引发自激振荡。

针对上述问题,在实际应用中通过在无信号状态下将超声波接收器输入端与地短接,
使运放输入保持稳定参考电位,从而有效抑制自激振荡现象。
此外,通过引入 $2.5\text{V}$ 偏置电压作为虚拟地,
使交流信号始终工作在单电源运放的线性输入范围内,并配合比较器电路对放大后的信号进行整形,
最终获得稳定、可靠的 $40\text{kHz}$ 数字脉冲信号,满足后续单片机时间差测量的需求。

\section{实验结果与分析}

实验结果表明,在超声波发送端持续发射信号的条件下,
小车能够在一定距离范围内稳定接收超声波信号,
并根据左右接收通道的时间差实现对声源方向的判断,从而完成自动跟随运动。
在实验过程中,小车在直行和转向过程中运行平稳,未出现明显的振荡或失控现象,
表明所采用的控制策略与参数设置具有较好的稳定性。

当声源位置发生变化时,小车能够通过调整左右电机转速差及时修正运动方向,
使接收时间差逐渐减小,最终保持正对声源的运动状态。
该过程验证了基于超声波时间差进行方向定位方法的可行性。

当超声波声源移出接收范围或信号暂时中断时,系统能够自动判定信号丢失并切换至搜索模式。
此时,小车通过单侧驱动方式进行缓慢转向扫描,从而扩大对声源的搜索范围。
当重新检测到有效超声波信号后,系统能够及时退出搜索状态并恢复正常跟随运动,
整体行为符合预期设计。

综合实验现象可以看出,系统在实际运行中对环境变化具有一定的适应能力,
能够在信号存在与消失的情况下完成状态切换,验证了系统软硬件协同设计的有效性。


\section{总结}

本文围绕基于超声波信号的自动跟随需求,设计并实现了一种超声波定位自动跟随小车系统。
系统采用分离式超声波发送与接收结构,通过双接收器时间差测量实现声源方向判断,
并结合嵌入式控制算法对小车运动进行实时调节。

在硬件方面,完成了超声波接收滤波与整形电路、电机驱动电路以及单片机控制系统的搭建;
在软件方面,采用模块化设计方法,实现了超声波信号检测、PID 方向控制、
电机差速驱动以及搜索与跟随状态切换等功能。实验结果表明,
该系统能够较为稳定地完成自动跟随与信号搜索任务,达到了实验预期目标。

通过本次实验,加深了对超声波信号处理、单电源模拟电路设计以及嵌入式系统控制策略的理解,
同时也提升了将硬件电路与软件算法相结合解决实际工程问题的能力。
本实验为进一步开展基于传感器的智能移动系统设计提供了有益的实践经验。


\section*{致谢}

在此向张老师和实验室的所有助教老师谨致以诚挚的感谢,也感谢实验室提供的良好实验条件和相关设备支持。

特别感谢张老师在实验方案设计、整体思路把握以及电气系统调试方面给予的指导与建议,使本实验在系统结构和实现路径上更加清晰合理。
也感谢助教老师在电路搭建、调试方法以及电机驱动模块的使用等方面提供的细致而及时的帮助,
有效解决了实验过程中遇到的多项实际问题,为实验的顺利完成提供了重要保障。

此外,感谢实验室提供的良好实验条件和相关设备支持。
本次实验不仅加深了我们对超声波应用、电机控制及嵌入式系统设计的理解,也提升了综合分析与工程实践能力。
在此一并表示衷心感谢。

\newpage

\appendix
\section{小车程序源代码}

\subsection{main.cpp}
\lstinputlisting[language=C++,caption={主程序 main.cpp}]{../Ultrasonic_Vehicle_2025/src/main.cpp}

\subsection{main\_controller.h}
\lstinputlisting[language=C++,caption={主控制器头文件 main\_controller.h}]{../Ultrasonic_Vehicle_2025/src/main_controller.h}

\subsection{motor\_controller.h}
\lstinputlisting[language=C++,caption={电机控制模块头文件 motor\_controller.h}]{../Ultrasonic_Vehicle_2025/src/motor_controller.h}

\subsection{ultrasonic\_controller.h}
\lstinputlisting[language=C++,caption={超声波检测模块头文件 ultrasonic\_controller.h}]{../Ultrasonic_Vehicle_2025/src/ultrasonic_controller.h}

\subsection{pid\_controller.h}
\lstinputlisting[language=C++,caption={PID 控制器头文件 pid\_controller.h}]{../Ultrasonic_Vehicle_2025/src/pid_controller.h}

\subsection{config.h}
\lstinputlisting[language=C++,caption={系统参数配置文件 config.h}]{../Ultrasonic_Vehicle_2025/src/config.h}

\section{超声波发生器程序源代码}

\lstinputlisting[language=C++,caption={主程序 main.cpp}]{../Ultrasonic_Transmitter/src/main.cpp}

\end{document}
